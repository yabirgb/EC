\documentclass{article}
\usepackage[left=3cm,right=3cm,top=2cm,bottom=2cm]{geometry} % page settings
\usepackage{amsmath} % provides many mathematical environments & tools
\usepackage{amssymb}
\usepackage{amsfonts}
\usepackage[spanish]{babel}



\usepackage{multirow}

\usepackage{algorithm}
\usepackage{algpseudocode}
\usepackage{pifont}

\usepackage[utf8]{inputenc}
\setlength{\parindent}{0mm}

\usepackage[parfill]{parskip}

% Para el código
\usepackage{listings}
\usepackage{xcolor}
\definecolor{gray}{rgb}{0.5,0.5,0.5}
\newcommand{\n}[1]{{\color{gray}#1}}
\lstset{numbers=left,numberstyle=\small\color{gray}}

% Entorno para estilo de ejercicios
\newenvironment{ejercicio}[1]{\textbf{#1} \vspace*{5mm}}{\vspace*{5mm}}
\setlength{\parindent}{10pt}

% Para marcar el codigo de ensamblador muy tocho pero he perdiido la referencia
\usepackage{listings}
\lstdefinelanguage[mips]{Assembler}{%
  % so listings can detect directives and register names
  alsoletter={.\$},
  % strings, characters, and comments
  morestring=[b]",
  morestring=[b]',
  morecomment=[l]\#,
  % instructions
  morekeywords={[1]abs,abs.d,abs.s,add,add.d,add.s,addi,addiu,addu,%
    and,andi,b,bc1f,bc1t,beq,beqz,bge,bgeu,bgez,bgezal,bgt,bgtu,%
    bgtz,ble,bleu,blez,blt,bltu,bltz,bltzal,bne,bnez,break,c.eq.d,%
    c.eq.s,c.le.d,c.le.s,c.lt.d,c.lt.s,ceil.w.d,ceil.w.s,clo,clz,%
    cvt.d.s,cvt.d.w,cvt.s.d,cvt.s.w,cvt.w.d,cvt.w.s,div,div.d,div.s,%
    divu,eret,floor.w.d,floor.w.s,j,jal,jalr,jr,l.d,l.s,la,lb,lbu,%
    ld,ldc1,lh,lhu,li,ll,lui,lw,lwc1,lwl,lwr,madd,maddu,mfc0,mfc1,%
    mfc1.d,mfhi,mflo,mov.d,mov.s,move,movf,movf.d,movf.s,movn,movn.d,%
    movn.s,movt,movt.d,movt.s,movz,movz.d,movz.s,msub,msubu,mtc0,mtc1,%
    mtc1.d,mthi,mtlo,mul,mul.d,mul.s,mulo,mulou,mult,multu,mulu,neg,%
    neg.d,neg.s,negu,nop,nor,not,or,ori,rem,remu,rol,ror,round.w.d,%
    round.w.s,s.d,s.s,sb,sc,sd,sdc1,seq,sge,sgeu,sgt,sgtu,sh,sle,%
    sleu,sll,sllv,slt,slti,sltiu,sltu,sne,sqrt.d,sqrt.s,sra,srav,srl,%
    srlv,sub,sub.d,sub.s,subi,subiu,subu,sw,swc1,swl,swr,syscall,teq,%
    teqi,tge,tgei,tgeiu,tgeu,tlt,tlti,tltiu,tltu,tne,tnei,trunc.w.d,%
    trunc.w.s,ulh,ulhu,ulw,ush,usw,xor,xori},
  % assembler directives
  morekeywords={[2].align,.ascii,.asciiz,.byte,.data,.double,.extern,%
    .float,.globl,.half,.kdata,.ktext,.set,.space,.text,.word},
  % register names
  morekeywords={[3]\$0,\$1,\$2,\$3,\$4,\$5,\$6,\$7,\$8,\$9,\$10,\$11,%
    \$12,\$13,\$14,\$15,\$16,\$17,\$18,\$19,\$20,\$21,\$22,\$23,\$24,%
    \$25,\$26,\$27,\$28,\$29,\$30,\$31,%
    \$zero,\$at,\$v0,\$v1,\$a0,\$a1,\$a2,\$a3,\$t0,\$t1,\$t2,\$t3,\$t4,
    \$t5,\$t6,\$t7,\$s0,\$s1,\$s2,\$s3,\$s4,\$s5,\$s6,\$s7,\$t8,\$t9,%
    \$k0,\$k1,\$gp,\$sp,\$fp,\$ra},
}[strings,comments,keywords]

%Para que el codigo se meta en el recuadro
\lstset{numbers=left,xleftmargin=2em,frame=single,framexleftmargin=1.5em}


\begin{document}

\title{Estructura de Computadores - Entrega 3}
\author{Yábir García Benchakhtir}
\date{\today}
\maketitle

\section{Tareas realizadas}

En esta práctica he construido y encontrado las claves de mi propia
bomba. Además he desactivado la bomba de mi compañera Patricia Cordoba
y la bomba que aparece en \textit{SWAD} con el nombre de \textit{NBA}.

\section{Bomba realizada}

Junto a este documento se acompaña la bomba realiza en \textit{C}. Los
datos para desactivarla son:

\begin{itemize}
  \item Contraseña: \textit{Q..W.r}
  \item Pin: \textit{8746}
\end{itemize}

\subsection{Pasos para la desactivación}

Usando \textit{gdb} abrimos el binario de nuestra bomba y ejecutamos
la orden \textit{disas} para comprobar las instrucciones de la bomba.

Si observamos el código vemos que se tras leer la entrada se comprueba
la longitud y después se llama a la función \textit{boom}.

\begin{lstlisting}
  0x56555808 <+135>:	call   0x56555560 <strlen@plt>
  0x5655580d <+140>:	add    $0x10,%esp
  0x56555810 <+143>:	cmp    %eax,%esi
  0x56555812 <+145>:	je     0x56555819 <main+152>
  0x56555814 <+147>:	call   0x565556dd <boom>
  0x56555819 <+152>:	movl   $0x0,-0x94(%ebp)

\end{lstlisting}

Si comprobamos el contenido de los registros comprobamos que solo esta
el tamaño de la contraseña que no es lo que buscabamos pero es
información relevante. En este caso sabemos que la contraseña tiene 6
caracteres.

Cerca de esta zona en el código tenemos una linea donde movemos
contendio de \textit{\%ebp} a \textit{\%eax}, nos hace sospechar que
pueda estar moviendo la contraseña. Añadiendo un break en esta linea e
imprimiento el contenido de los registros obtenemos:

\newpage

\begin{lstlisting}
  (gdb) info reg
  eax            0x0	0
  ecx            0x8	8
  edx            0x56557008	1448439816
  ebx            0x56556fb8	1448439736
  esp            0xffffced0	0xffffced0
  ebp            0xffffcf68	0xffffcf68
  esi            0x7	7
  edi            0xf7fb1000	-134541312
  eip            0x56555831	0x56555831 <main+176>
  eflags         0x202	[ IF ]
  cs             0x23	35
  ss             0x2b	43
  ds             0x2b	43
  es             0x2b	43
  fs             0x0	0
  gs             0x63	99
  (gdb) x/s \$edx
  0x56557008 <password>:	"Q..W.r\n"

\end{lstlisting}
 
De esta manera ya conocemos la contraseña de la bomba.
Consultando de nuevo el código del programa observamos la linea:

\begin{lstlisting}
  0x565558c3 <+322>: call 0x56555580 <__isoc99_scanf@plt>
\end{lstlisting}

donde realiza la lectura desde el teclado del código. Mirando cerca de
esta instrucción observamos como en el caso de la contraseña:

\begin{lstlisting}
  => 0x565558d1 <+336>:	mov    0x58(\%ebx),\%eax
\end{lstlisting}

Si imprimimos el contenido de los registros obtenemos:

\begin{lstlisting}
  eax            0x222a	8746
  ecx            0x1	1
  edx            0x913	2323
  ebx            0x56556fb8	1448439736
  esp            0xffffced0	0xffffced0
  ebp            0xffffcf68	0xffffcf68
  esi            0x7	7
  edi            0xf7fb1000	-134541312
  eip            0x565558d7	0x565558d7 <main+342>
  eflags         0x282	[ SF IF ]
  cs             0x23	35
  ss             0x2b	43
  ds             0x2b	43
  es             0x2b	43
  fs             0x0	0
  gs             0x63	99
\end{lstlisting}

donde en \textit{eax} tenemos el código de la bomba.

\section{Bomba NBA}

Usando la herramienta \textit{ghex} comprobamos que hay tres cadenas
de texto que pueden ser la contraseña pero no sabemos cual de ellas
puede ser. En concreto son: \textit{Oh, castitas lilium}, \textit{Esta
  es la clave!!} y \textit{Miauuu}.

En la dirección \textit{0x08048806} mueve la contraseña a los
registros en este caso es \textit{Oh, castitas lilium}. Las otras dos
cadenas las compara en las direcciones \textit{0x804b030} y
\textit{0x080487ac}. En el caso de las otras dos cadenas de ser una de
ellas llama a la función \textit{<f1\_ew>} que nos muestra en pantalla
\textit{miau}.

Observando el código vemos:

\begin{lstlisting}
     0x08048880 <+575>:	mov    0x804b064,\%eax
     0x08048885 <+580>:	cmp    \%eax,\%edx
\end{lstlisting}

si en esta dirección imprimos los registros vemos:

\begin{lstlisting}
  eax            0x406	1030
  ecx            0x1	1
  edx            0x406	8888
  ebx            0x0	0
  esp            0xffffcec0	0xffffcec0
  ebp            0xffffcf68	0xffffcf68
  esi            0x1	1
  edi            0xf7fb1000	-134541312
  eip            0x8048885	0x8048885 <main+580>
  eflags         0x246	[ PF ZF IF ]
  cs             0x23	35
  ss             0x2b	43
  ds             0x2b	43
  es             0x2b	43
  fs             0x0	0
  gs             0x63	99

\end{lstlisting}

donde en \textit{eax} tenemos el pin y en \textit{edx} está el que
hemos metido nostros.
   
\section{Bomba de Patricia Cordoba}

Los datos de esta bomba son:

\begin{itemize}
  \item Contraseña: \textit{t...t...}
  \item Pin: \textit{4444}
\end{itemize}

Los pasos de manera esquematica han sido:

\begin{itemize}
\item Añadir break en main.
\item Avanzar hasta que nos pida la contraseña.
\item En la instrucción \textit{0x0804877f} vemos un push extraño
  cerca de donde hemos introducido la contraseña. Si imprimimos su contenido vemos:

  \begin{lstlisting}
    (gdb) x/s 0x804a0a5
    0x804a0a5 <lalala>: "t...t...\n"
  \end{lstlisting}


\item En la instrucción \textit{0x0804884d} ya hemos introducido el codigo.
\item Vemos una comparación en \textit{0x08048853}.
\item Vemos que accede a una posición con aritemética de punteros.
\item Hacemos \textit{p \*(int \*)(\$ebp-140)} 140 se corresponde con
  0x8c. Así obtenemos el código de la bomba, en este caso, 4444.
\end{itemize}
\end{document}