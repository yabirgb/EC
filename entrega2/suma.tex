\section*{Preguntas de autocombración: suma\_08\_Casm}

\begin{ejercicio}{1. Obtener el código de ensamblador generado con gcc
    y compararlo con el anterior de \textit{suma\_08} ¿Qué diferencias
    hay?}

  Como principal diferencia encontramos que el código generado por ensamblador combrueba que la lista sea no vacía. También podemos destacar que en lugar de usar \textbf{inc} utiliza \textbf{addl}.
\end{ejercicio}

\begin{ejercicio}{2. Comparar el cód digo generado comentando
    \textit{cc} y descomentando \textit{cc} de la lista clobber.  ¿Hay
    alguna diferencia?}
  
  Haciendo la compilación con la orden \textit{gcc -S suma\_08\_Casm.c
    -o suma8.s -m32 -fno-omit-frame-pointe} y usando la orden
  \textit{diff} no hay ninguna diferencia.
\end{ejercicio}

\begin{ejercicio}{3. No necesitamos declarar ningún otro sobreescrito, pero por un motivo distinto que en el ejemplo anterior. ¿Por qué?}

  No necesitamos ninguna restricción extra porque los registros están
  puestos por gcc. Lo hacemos así ya que necesitamos acceder a los
  valores proporcionados a la función.
  
\end{ejercicio}

\begin{ejercicio}{4. Si res es una variable de salida ¿por qué se le ha
    indicado restricción \textit{+r} en lugar de \textit{=r}?}

  Necesitamos marcarlo con la opción de escritura ya que al usar add
  estamos modificando su valor.
  
\end{ejercicio}

\begin{ejercicio}{5. Explicar por qué en este caso se prefiere acabar la linea en \textit{\textbackslash n} en lugar de \textit{\textbackslash n \textbackslash t}}

  Cuando usamos \textit{(\textbackslash n \textbackslash t} lo hacemos para indentar la siguiente
  linea e indicar estructuras que dependen de otras. En este caso
  tenemos una única linea y las instrucción siguiente no van a
  depender de eta.
  
\end{ejercicio}

\section*{Preguntas de autocombración: suma\_09\_Casm}

\begin{ejercicio}{1. Repasar el código de ensamblador generado por gcc
    para las tres versiones. ¿Hay alguna diferencia?}

  En \textit{suma\_09} la función suma2 se corresponde con la de suma\_08
  y la función suma3 con la de suma\_07. Sus códigos ensambladores son
  casi identicos. 
  
\end{ejercicio}

\begin{ejercicio}{2. En la versión 3 se ha añadido un clobber que antes
    no estaba (ver Figura 10). ¿Acaso no sirve para nada ese clobber?
    ¿No hay diferencias en el código en ensamblador generado?}

  La diferencia más notoria es que en suma\_09 se hace \textit{pushl
    \% ebx}. Con lo cual está salvando su contenido y por lo tanto se
  está preocupando de salvar el contenido de cara a otras funciones
  que puedan usar su contenido.
  
\end{ejercicio}

\begin{ejercicio}{3. En la versión 3 se han escrito los registros
    con dos símbolos \%, en lugar de uno (como en la Figura 10).
    ¿Qué pasa si se escriben como antes? ¿Por qué no pasaba eso
    antes?}

  Escribimos ahora doble porcentaje porque en los clobber hemos puesto
  \textit{ebx} y para indicarle a gcc que nos referimos al registro y
  no al clobber ponemos el doble porcentaje.
  
\end{ejercicio}

\begin{ejercicio}{4. ¿Cuántos elementos tiene el array? ¿Cuánta memoria
    ocupa? ¿Cuánto vale la suma? ¿Cómo se llaman ese tipo de sumas?
    ¿Qué fórmula se usa para calcular una suma como esa?}

  Como estamos desplazando un \textit{1} 16 posiciones a la izquiera
  el tamaño del array será de $2^{16}$. Como lo estamos definiendo
  como un array de \textit{int} y suponiendo que este tipo de dato usa
  $4B$ el tamaño del array será $4*2^{16} = 2^{18}$. El valor de la suma es 2147516416 y se calcula sabiendo que se trata de una progresión aritmética:

  $$\sum_{i=0}^{2^{16}}i = \frac{n(a_1 + a_n)}{2}$$

  donde $n$ es el número total de términos, $a_1$ es el primer término
  de la sucesión y $a_n$ es el último término.
  
\end{ejercicio}

\begin{ejercicio}{5. El código C imprime un mensaje diciendo
    $\frac{N\cdot (N+1)}{2}$, pero luego calcula $\frac{(SIZE-1)\cdot
      SIZE}{2}$. ¿Cuál es la fórmula correcta?}

  Ambas son correctas porque en el caso de que este lleno $SIZE = N+1$
  y usando la igualdad del ejercicio anterior ambas formulas son
  iguales.
  
\end{ejercicio}

\newpage

\begin{ejercicio}{6. Esa línea viene comentada con /* OF */. ¿Qué
    puede significar ese comentario?  ¿Qué se puede decir acerca
    de la forma de escribir esa fórrmula? Si es por ``incomodida
    para calcular la fórmula'', ¿qué se podría haber hecho para evitar
    de golpe cualquier incomodidad? ¿Cómo se escribiría
    entonces, más cómodamente, la fórmula, y tooda la instrucción
    printf?}

  Puede significar que se produzca un overflow ya que estamos
  calculando un número que es del orden de $2^{32}$. Para solucionar
  esto podemos usar tipos de datos que permitan almacenar un mayor
  tamaño. Podríamos haber escrito la orden printf como:\\

  \textit{printf("\%lu \textbackslash n", SIZE*(SIZE-1)/2)}
  
\end{ejercicio}

\begin{ejercicio}{8. ¿Hay alguna manera de ganar a gcc haciendo el
    tipo de cosas que venimos haciendo con suma?}

  Por lo general no, aunque usando asm podemos obtener mejor
  rendimiento en algunos casos o cuando trabajamos con sistemas
  especiales que cuenten con un repertorio de instrucciones que el
  compilador no sepa manejar o no de manera optima.
\end{ejercicio}